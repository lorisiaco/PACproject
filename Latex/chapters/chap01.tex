\chapter{Iterazione 0}
\section{Introduzione}
Spendly è una web-app innovativa progettata per semplificare la gestione delle spese personali e condivise. Con un’interfaccia intuitiva e funzionalità avanzate, Spendly consente agli utenti di tracciare le proprie spese quotidiane, mantenendo al contempo il controllo sui budget di gruppo. Che si tratti di viaggi, cene, o progetti collaborativi, Spendly elimina la complessità nel calcolo dei debiti reciproci grazie all'implementazione dell'algoritmo di Dijkstra.
\\
\\
L'app ottimizza la distribuzione dei saldi all'interno dei gruppi, trovando il percorso più efficiente per minimizzare le transazioni tra i membri. Spendly non è solo un’app per il monitoraggio delle finanze, ma uno strumento che rende più semplice e trasparente la collaborazione economica tra persone.
\\
\\
Con Spendly, dire addio ai calcoli manuali complicati è semplice. Che si tratti di un gruppo di amici in vacanza, colleghi che condividono un progetto, o coinquilini che dividono le bollette, Spendly garantisce una soluzione chiara, equa e immediata.
\\
\begin{itemize}
    \item \textbf{Esaminazione dello schema elettrico} : il circuito del microgrid è un circuito relativamente semplice. Una volta studiato si applicano le leggi di Kirchhoff ai nodi per ricavare le equazioni differenziali, affinchè successivamente si possa sviluppare il sistema dinamico. 
    \item \textbf{Ricavo del sistema dinamico dallo schema elettrico }: una volta scelte le variabili x(t), y(t), u(t), e la d(t), rispettivamente le variabili di stato, l'uscita, la variabile di controllo ed il disturbo, si procede alla stesura del sistema di equazioni differenziali. \\Facendo così si otterrà la funzione di trasferimento su cui si potrà lavorare per la determinazione del migliore controllore PID.
    \item \textbf{Taratura del controllore PID}: si useranno 4 metodi di taratura:\begin{enumerate}
        \item Metodo di Ziegler-Nichols(Z-G)
        \item Metodo di Chien-Hrones-Reswick(CHR)
        \item Metodo Cohen-Coon(C-C)
        \item Metodo Wang-Juang-Chan(WJC)
    \end{enumerate}
    \item \textbf{Analisi dei risultati dei metodi di taratura}: verranno esaminati e comparati i vari valori ottenuti con i 4 metodi di taratura per la selezione del migliore, che rispetti al meglio le specifiche desiderate.
\end{itemize}

