\documentclass{softwaredoc}

\title{Spendly: Documentazione Tecnica}
\author{Amin Borqal, Loris Iacoban, Diego Rossi}
\date{\today}

\begin{document}

% ======================
% FRONTESPIZIO
% ======================
\maketitle
\newpage % Pagina vuota dopo il titolo
\thispagestyle{empty} % Rimuove numerazione pagina
\mbox{} % Pagina bianca
\newpage 

% ======================
% INDICI
% ======================
\tableofcontents
\addtocontents{toc}{\protect\thispagestyle{plain}} % Forza la pagina con stile normale
\newpage
\listoffigures
\listoftables
\newpage 
% ======================
% ITERAZIONE 0
% ======================
\section*{Iterazione 0}
Questa sezione analizza i requisiti iniziali del progetto Spendly.

\subsection{Introduzione}

In questa iterazione, abbiamo implementato la gestione del Budget, permettendo agli utenti di monitorare e gestire i propri fondi disponibili. Inoltre, abbiamo introdotto la funzionalità di Risparmio, che consente agli utenti di creare obiettivi di risparmio per mettere da parte denaro per esigenze future.

\newpage
\subsection{Requisiti}
Il sistema deve soddisfare i seguenti requisiti funzionali e non funzionali:

\begin{itemize}
    \item \textbf{Requisiti Funzionali}:
    \begin{itemize}
        \item Gli utenti devono poter registrarsi e autenticarsi al sistema.
        \item Gli utenti devono poter aggiungere, modificare e cancellare le proprie spese.
        \item Gli utenti devono poter creare gruppi e gestire le spese condivise.
        \item Il sistema deve calcolare automaticamente il bilancio dei debiti tra i membri del gruppo.
        \item Il sistema deve inviare notifiche o alert agli utenti quando il loro budget per una determinata categoria di spesa viene superato.
        \item Gli utenti devono poter impostare budget personalizzati per categorie specifiche (es. alimentari, trasporti, svago, ecc.).
    \end{itemize}
    
    \item \textbf{Requisiti Non Funzionali}:
    \begin{itemize}
        \item L'applicazione deve essere intuitiva e facile da usare.
        \item Il sistema deve garantire la sicurezza dei dati personali degli utenti mediante crittografia e autenticazione sicura.
        \item Le transazioni e le modifiche devono essere sincronizzate in tempo reale tra tutti i dispositivi degli utenti.
        \item L’applicazione deve essere accessibile da diversi dispositivi (smartphone, tablet, desktop).
        \item Il sistema deve garantire elevate prestazioni, assicurando una risposta rapida alle richieste degli utenti.
    \end{itemize}
\end{itemize}

\newpage
\subsection{Casi d'Uso}

    \begin{figure}[h]
        \centering
        \includegraphics[width=1.1\textwidth, trim=3cm 0cm 1cm 0cm]{images/DiagrammaCasiDusoV1.2.png}
        \caption{Diagramma casi d'uso }
    \end{figure}

Di seguito sono riportati i casi d'uso in figura:
\begin{itemize}
    \item \textbf{UC1 - Login}: Come utente, voglio poter accedere al mio account per gestire le mie spese.
    \item \textbf{UC2 - Registrazione}: Come nuovo utente, voglio potermi registrare al sistema per iniziare a usare la web-app.
    \item \textbf{UC3 - Logout}: Voglio poter terminare la mia sessione.
    \item \textbf{UC4 - Crea alert di gruppo}: Come utente amministratore di un gruppo spesa voglio poter inserire un alert, dove un alert è un avviso che ci permette di avvisare se si sta raggiungendo una soglia limite di spesa.
    \item \textbf{UC5 - Modifica alert}: Voglio poter modificare i valori dell'alert.
    \item \textbf{UC6 - Elimina alert}: Voglio poter eliminare l'alert. 
    \item \textbf{UC7 - Crea gruppo}: Voglio poter creare un gruppo di condivisione spese. 
    \item \textbf{UC8 - Invita memebri}: Voglio come amministratore invitare membri nel gruppo di spese.
    \item \textbf{UC9 - Elimina membri}: Voglio come amministratore poter eliminare membri del gruppo di spese.
    \item \textbf{UC10 - Modifica membri}: Voglio come amministratore modificare i membri nel gruppo di spese.
    \item \textbf{UC11 - Elimina gruppo}: Voglio poter eliminare un gruppo di condivisione spese. 
    \item \textbf{UC12 - Accedi gruppo}: Voglio poter accedere ad un gruppo di condivisione spese. 
    \item \textbf{UC13 - Inserisci spesa}: Voglio poter inserire una spesa di gruppo.
    \item \textbf{UC14 - Elimina spesa}: Voglio poter eliminare una spesa di gruppo.
    \item \textbf{UC15 - Modifica spesa}: Voglio poter modificare una spesa di gruppo.
    \item \textbf{UC16 - scegli tipologia spesa}: Voglio poter scegliere la tipologia di spesa.
    \item \textbf{UC17- Visualizza spese}: Voglio poter visualizzare le spesa di gruppo.
    \item \textbf{UC18 - Ricalcolo debiti}: Voglio poter calcolare i debiti.
    \item \textbf{UC19 - Inserisci spesa personale}: Voglio poter inserire una spesa personali.
    \item \textbf{UC20 - Elimina spesa personale}: Voglio poter eliminare una spesa personali.
    \item \textbf{UC21 - Modifica spesa personale}: Voglio poter modificare una spesa personali.
    \item \textbf{UC22 - scegli tipologia spesa personale}: Voglio poter scegliere la tipologia di spesa personale.
    \item \textbf{UC23 - Visualizza spesa personale}: Voglio poter visualizzare le spesa personali.
    \item \textbf{UC24 - Aggiungi denaro a budget}: Voglio poter inserire un budget di risparmio.
    \item \textbf{UC25 - Togli denaro da budget}: Voglio poter eliminare un budget di risparmio.
    \item \textbf{UC26 - Visualizza budget}: Voglio poter visualizzare i budget di risparmio.
\end{itemize}


\subsection{Architettura del Sistema}

Il sistema segue il pattern architetturale \textbf{Model-View-Controller (MVC)}, un'architettura software che separa la logica di presentazione, la logica di business e la gestione dei dati. Questa separazione consente una maggiore modularità e facilita la manutenzione e l'estendibilità del sistema. Le tre componenti principali sono:
\begin{figure}[H]
    \centering
    \includegraphics[scale=0.45]{images/mvc.png}
    \caption{M-V-C Pattern }
\end{figure}
\subsubsection{Model}
Il \textbf{Model} rappresenta la logica di business e la gestione dei dati. In questo sistema:
\begin{itemize}
    \item I dati sono memorizzati in un database relazionale \textbf{PostgreSQL}, scelto per la sua affidabilità e scalabilità.
    \item Il backend, implementato in \textbf{Spring Boot}, gestisce le operazioni sui dati attraverso un'architettura basata su servizi.
    \item I principali servizi includono:
    \begin{itemize}
        \item \textbf{Servizio Utente}: gestione della registrazione, autenticazione (tramite \textbf{JWT}) e gestione del profilo utente.
        \item \textbf{Servizio Gestione Spese}: esecuzione delle operazioni CRUD sulle spese degli utenti.
        \item \textbf{Servizio Gruppi}: creazione e gestione dei gruppi di utenti.
    \end{itemize}
    \item L'interazione con il database avviene tramite il \textbf{Data Access Layer}, basato su JPA e Hibernate.
\end{itemize}

\subsubsection{View}
La \textbf{View} è responsabile della presentazione dei dati all'utente e della gestione dell'interazione con il sistema. Nel nostro caso:
\begin{itemize}
    \item Il frontend è sviluppato utilizzando \textbf{Vue.js}, un framework JavaScript progressivo che permette lo sviluppo di interfacce utente reattive e modulari.
    \item Le chiamate API al backend vengono gestite tramite \textbf{Axios}, consentendo un flusso di dati asincrono tra client e server.
    \item L'autenticazione degli utenti è gestita tramite token \textbf{JWT}, che vengono memorizzati e inviati nelle richieste HTTP per garantire la sicurezza.
\end{itemize}

\subsubsection{Controller}
Il \textbf{Controller} funge da intermediario tra il Model e la View, gestendo le richieste utente e applicando la logica di business. In questo sistema:
\begin{itemize}
    \item Il backend espone un'API \textbf{RESTful} tramite Spring Boot, con endpoint per la gestione di utenti, spese e gruppi.
    \item Ogni richiesta HTTP viene elaborata da un controller dedicato, che interagisce con i servizi del Model per recuperare o aggiornare i dati.
    \item Le risposte sono restituite in formato JSON, consentendo una comunicazione fluida con il frontend.
\end{itemize}

\subsubsection{Flusso delle Richieste}
Il flusso di una tipica richiesta utente segue questi passi:
\begin{enumerate}
    \item L'utente interagisce con il frontend Vue.js tramite un'interfaccia web.
    \item Un'azione (es. login, aggiunta di una spesa) genera una richiesta HTTP verso il backend.
    \item Il \textbf{Controller} riceve la richiesta e invoca il servizio appropriato.
    \item Il \textbf{Model} esegue la logica di business e interagisce con il database per recuperare o aggiornare i dati.
    \item La risposta viene inviata al frontend in formato JSON.
    \item Il frontend aggiorna dinamicamente l'interfaccia utente con i nuovi dati.
\end{enumerate}
Questa architettura garantisce una chiara separazione delle responsabilità, facilitando lo sviluppo, la scalabilità e la manutenzione del sistema.


\subsection{Priorità casi d'uso}
I casi d'uso sono suddivisi in tre categorie, in linea con l'ordine stabilito in \textbf{Casi d'Uso}:

\begin{itemize}
    \item \textbf{Coda ad alta priorità:} 
    \begin{itemize}
        \item UC1 - Login
        \item UC2 - Registrazione
        \item UC3 - Logout
        \item UC7 - Crea gruppo
        \item UC8 - Invita membri
        \item UC9 - Elimina membri
        \item UC10 - Modifica membri
        \item UC11 - Elimina gruppo
        \item UC12 - Accedi gruppo
        \item UC13 - Inserisci spesa
        \item UC14 - Elimina spesa
        \item UC15 - Modifica spesa
        \item UC16 - Scegli tipologia spesa
        \item UC17 - Visualizza spese
        \item UC18 - Ricalcolo debiti
    \end{itemize}
    
    \item \textbf{Coda a media priorità:} 
    \begin{itemize}
        \item UC4 - Crea alert di gruppo
        \item UC5 - Modifica alert
        \item UC6 - Elimina alert
        \item UC19 - Inserisci spesa personale
        \item UC20 - Elimina spesa personale
        \item UC21 - Modifica spesa personale
        \item UC22 - Scegli tipologia spesa personale
        \item UC23 - Paga spesa personale
        \item UC24 - Visualizza spesa personale
    \end{itemize}
    
    \item \textbf{Coda a bassa priorità:} 
    \begin{itemize}
        \item UC25 - Aggiungi denaro a budget
        \item UC26 - Togli denaro da budget
        \item UC27 - Visualizza budget
    \end{itemize}
\end{itemize}

\textbf{Perché si dividono in code di priorità i casi d'uso?}
\\
\\
In quanto lo sviluppo del progetto viene sviluppato tramite il processo Agile Model-Driven Developmen.
\\
AMDD combina i principi della modellazione guidata dallo sviluppo con la flessibilità dei metodi agili, che a differenza dello sviluppo tradizionale, in cui tutta l’analisi viene completata prima di iniziare la programmazione, prevede una modellazione iniziale leggera e iterativa, concentrata solo sugli aspetti essenziali per l’iterazione corrente.
\newline
Ad ogni ciclo di sviluppo, il team seleziona un numero limitato di casi d’uso dalla coda di priorità più alta, lavorando su di essi fino a ottenere una funzionalità completa e testata. Questo approccio:
\begin{itemize}
    \item Riduce il rischio affrontando le funzionalità critiche prima possibile.
    \item Favorisce il feedback continuo, permettendo agli stakeholder di validare il progresso e suggerire miglioramenti.
    \item Adatta lo sviluppo alle necessità reali, permettendo di rivalutare le priorità in base all’andamento del progetto.
\end{itemize}
Grazie a questo metodo, il sistema viene costruito in modo solido e incrementale, con la possibilità di aggiungere nuove funzionalità man mano che il progetto evolve.
\newpage
\subsection{Topologia del Sistema}

La topologia del sistema è rappresentata nel seguente schema:

\begin{figure}[h]
\centering
\includegraphics[scale=0.75]{images/topologia.png}
\caption{Diagramma dell’architettura della webapp Spendly}
\end{figure}

Il sistema è basato unicamente su un’architettura \textbf{web}, in cui i client interagiscono con un \textbf{web server} sviluppato in \textbf{Spring Boot}. I client utilizzano un browser, comunicando con il server attraverso \textbf{REST-API HTTP}, con dati in formato \textbf{JSON}. 

\subsubsection{Protocollo HTTP e REST-API}

HTTP è un protocollo di trasferimento ipertestuale che offre un meccanismo semplice e universale per inviare e ricevere dati. È ideale per le nostre REST-API poiché assicura la compatibilità con i browser e l’interscambio di dati in JSON su qualsiasi piattaforma.

\subsubsection{Caratteristiche dell’architettura}

\begin{itemize}
\item \textbf{Scalabilità}: possibilità di scalare l’intero sistema o le singole funzionalità in base alle necessità.
\item \textbf{Modularità}: sviluppo e manutenzione semplificati grazie all’organizzazione delle funzionalità in moduli indipendenti.
\item \textbf{Affidabilità}: eventuali malfunzionamenti in una funzionalità non compromettono l’intero sistema.
\item \textbf{Manutenibilità}: possibilità di aggiornare o sostituire singoli moduli senza interrompere l’operatività complessiva.
\end{itemize}
\newpage
\subsection{Strumenti Utilizzati}

Lo sviluppo del progetto è supportato da una serie di strumenti software che facilitano la progettazione, lo sviluppo, il testing e la gestione del codice. Di seguito, una panoramica dettagliata dei tool impiegati:

\begin{itemize}
    \item \textbf{Visual Studio Code}: IDE (Integrated Development Environment) scelto per lo sviluppo del progetto. VS Code offre un ambiente di sviluppo leggero ma potente, con un ampio supporto per il linguaggio Java e numerose estensioni utili. Tra i plugin utilizzati troviamo:
    \begin{itemize}
        \item \textbf{Spring Boot Extension Pack}: per il supporto avanzato nello sviluppo di applicazioni Spring Boot.
        \item \textbf{Java Extension Pack}: per il supporto completo al linguaggio Java, con funzionalità di autocompletamento, refactoring e debugging.
    \end{itemize}
    
    \item \textbf{Spring Boot}: Framework Java basato su Spring, progettato per lo sviluppo di applicazioni web scalabili e strutturate secondo l’architettura a microservizi. Grazie alla sua configurazione automatica e al supporto integrato per REST API, Spring Boot semplifica la gestione del backend e garantisce un’elevata manutenibilità del codice.
    
    \item \textbf{PostgreSQL}: Sistema di gestione di database relazionale (RDBMS) scelto per la sua affidabilità, scalabilità e aderenza agli standard SQL. PostgreSQL supporta transazioni ACID (Atomicità, Coerenza, Isolamento, Durabilità) ed è ottimizzato per operazioni complesse e interrogazioni avanzate. Il database è stato configurato per garantire prestazioni elevate e sicurezza dei dati.

    \item \textbf{Postman}: Strumento essenziale per il testing delle API REST sviluppate con Spring Boot. Consente di effettuare richieste HTTP, validare le risposte del server e automatizzare test, facilitando così il debug e l’integrazione tra i diversi componenti del sistema.

    \item \textbf{Grok AI}: Tecnologia avanzata per la generazione automatica di immagini basata su intelligenza artificiale. Grok AI viene utilizzato per creare rappresentazioni visive intuitive e schematiche di concetti complessi, supportando la documentazione e la comunicazione grafica del progetto.

    \item \textbf{JAutoDoc}: Plugin per la generazione automatica della documentazione Javadoc a partire dai commenti presenti nel codice sorgente. Questo strumento semplifica la creazione di documentazione chiara e dettagliata per ogni componente del sistema.

    \item \textbf{JUnit 4}: Framework per il testing unitario in Java, impiegato per validare il corretto funzionamento delle classi e dei metodi sviluppati. L’uso di test automatizzati consente di rilevare tempestivamente eventuali errori e di garantire la robustezza del codice.

    \item \textbf{EclEmma}: Plugin per la misurazione della copertura del codice nei test unitari. Fornisce statistiche dettagliate che permettono di individuare porzioni di codice non coperte dai test, migliorando così la qualità complessiva del software.

    \item \textbf{JGraphT}: Libreria Java dedicata alla modellazione e alla gestione di strutture dati basate su grafi. Utilizzata per la rappresentazione e la manipolazione di relazioni complesse all’interno del sistema.

    \item \textbf{STAN4J}: Software per l'analisi statica del codice Java, impiegato per individuare problemi di progettazione, dipendenze indesiderate e violazioni dei principi di modularità.

    \item \textbf{GitHub}: Piattaforma per il versionamento del codice basata su Git, utilizzata per il controllo delle versioni e la collaborazione tra gli sviluppatori. Grazie a GitHub, è possibile tracciare le modifiche al codice, gestire le revisioni e garantire un workflow ordinato ed efficiente. 

    \item \textbf{GitHub Desktop}: Applicazione con interfaccia grafica che semplifica l’interazione con il repository GitHub direttamente dal PC. Permette di eseguire operazioni di commit, push e pull senza dover utilizzare la riga di comando, facilitando la gestione del codice per gli sviluppatori.

    \item \textbf{StarUML}: Software utilizzato per la modellazione di diagrammi UML (Unified Modeling Language), fondamentale nella fase di progettazione dell'architettura del sistema. StarUML consente di rappresentare visivamente classi, casi d’uso e flussi operativi.

    \item \textbf{diagrams.net}: Applicazione web per la creazione di diagrammi e schemi con notazione libera. Utilizzata per rappresentare graficamente flussi di dati, processi e architetture software, facilitando la comprensione e la condivisione delle informazioni progettuali.

    \item \textbf{WhatsApp}: Applicazione di messaggistica utilizzata come strumento di comunicazione interna tra i membri del team. Attraverso WhatsApp, è possibile coordinare le attività di sviluppo, discutere problemi tecnici e organizzare riunioni in tempo reale, garantendo un flusso comunicativo rapido ed efficiente.

\end{itemize}



% ======================
% ITERAZIONE 1
% ======================
\newpage
\section*{Iterazione 1}
In questa iterazione, ci concentriamo sul design e sull'implementazione iniziale del sistema.

\subsection{Casi d'Uso Implementati }

In questa iterazione sono stati sviluppati i seguenti casi d’uso ritenuti prioritari per lo sviluppo di \textbf{Spendly}.

\subsubsection{UC1: Login}
\textbf{Attori}: Utente, Sistema.

\textbf{Descrizione}: L'utente può autenticarsi nel sistema per accedere al proprio account.

\textbf{Flusso degli eventi}:
\begin{enumerate}
    \item L'utente accede alla pagina di login.
    \item Inserisce email e password.
    \item Il sistema verifica le credenziali e autentica l'utente.
    \item L'utente viene reindirizzato alla dashboard.
\end{enumerate}

\subsubsection{UC2: Registrazione}
\textbf{Attori}: Utente, Sistema.

\textbf{Descrizione}: Un nuovo utente può registrarsi creando un account.

\textbf{Flusso degli eventi}:
\begin{enumerate}
    \item L'utente accede alla pagina di registrazione.
    \item Inserisce nome, email e password.
    \item Il sistema verifica che l’email non sia già registrata.
    \item Se la verifica è superata, il sistema crea l’account e lo memorizza.
    \item L'utente viene reindirizzato alla dashboard.
\end{enumerate}

\subsubsection{UC3: Logout}
\textbf{Attori}: Utente, Sistema.

\textbf{Descrizione}: L'utente può terminare la sessione ed effettuare il logout.

\textbf{Flusso degli eventi}:
\begin{enumerate}
    \item L'utente clicca su "Logout".
    \item Il sistema invalida la sessione e mostra la schermata di login.
\end{enumerate}

\subsubsection{UC7: Creazione Gruppo}
\textbf{Attori}: Utente amministratore, Sistema.

\textbf{Descrizione}: L’utente può creare un nuovo gruppo di spese per la condivisione con altri membri.

\textbf{Flusso degli eventi}:
\begin{enumerate}
    \item L’utente clicca su "Crea Gruppo".
    \item Inserisce il nome del gruppo e una descrizione opzionale.
    \item Il sistema crea il gruppo e assegna l’utente come amministratore.
    \item L'utente viene reindirizzato alla pagina del gruppo.
\end{enumerate}

\subsubsection{UC8: Invita Membri}
\textbf{Attori}: Utente amministratore, Sistema.

\textbf{Descrizione}: L’amministratore di un gruppo può invitare altri utenti a unirsi al gruppo di spese.

\textbf{Flusso degli eventi}:
\begin{enumerate}
    \item L’amministratore accede alla pagina del gruppo.
    \item Clicca su "Invita Membri" e inserisce l’email degli utenti da invitare.
    \item Il sistema invia un’email con l’invito e memorizza la richiesta.
    \item Gli utenti ricevono l’invito e possono accettarlo per entrare nel gruppo.
\end{enumerate}

\subsubsection{UC9: Elimina Membri}
\textbf{Attori}: Utente amministratore, Sistema.

\textbf{Descrizione}: L’amministratore di un gruppo può rimuovere un membro dal gruppo.

\textbf{Flusso degli eventi}:
\begin{enumerate}
    \item L’amministratore accede alla lista dei membri del gruppo.
    \item Seleziona il membro da rimuovere e clicca su "Elimina".
    \item Il sistema rimuove il membro e aggiorna la lista.
\end{enumerate}

\subsubsection{UC10: Modifica Membri}
\textbf{Attori}: Utente amministratore, Sistema.

\textbf{Descrizione}: L’amministratore di un gruppo può modificare i dettagli dei membri (ad esempio assegnare nuovi ruoli).

\textbf{Flusso degli eventi}:
\begin{enumerate}
    \item L’amministratore accede alla lista dei membri.
    \item Seleziona un membro e modifica i dettagli (es. ruolo nel gruppo).
    \item Il sistema aggiorna i dati e notifica il cambiamento.
\end{enumerate}

\subsubsection{UC11: Eliminazione Gruppo}
\textbf{Attori}: Utente amministratore, Sistema.

\textbf{Descrizione}: L’amministratore di un gruppo può eliminare definitivamente un gruppo di spese.

\textbf{Flusso degli eventi}:
\begin{enumerate}
    \item L’amministratore accede alle impostazioni del gruppo.
    \item Clicca su "Elimina Gruppo".
    \item Il sistema chiede conferma prima di procedere.
    \item Se confermato, il gruppo e tutte le sue spese vengono eliminate.
\end{enumerate}

\subsubsection{UC12: Accesso a un Gruppo}
\textbf{Attori}: Utente, Sistema.

\textbf{Descrizione}: Un utente può accedere a un gruppo di spese a cui è stato invitato.

\textbf{Flusso degli eventi}:
\begin{enumerate}
    \item L'utente riceve un invito via email o notifica nell’app.
    \item Clicca sul link di invito e accede alla web-app.
    \item Il sistema verifica la validità dell’invito e aggiunge l’utente al gruppo.
    \item L'utente viene reindirizzato alla pagina del gruppo.
\end{enumerate}
\newpage
\subsection{UML Component Diagram}

\begin{center}
    \includegraphics[scale=0.45]{images/ComponentDiagramV1.3.png}
\end{center}

In questa iterazione, aggiorniamo il diagramma dei componenti rispetto all'iterazione 1 data l'introduzione dei costi. 
Partendo dai casi d’uso selezionati per questa iterazione e procedendo con l’utilizzo delle euristiche di design, è stato possibile progettare l’architettura software del sistema \textbf{Spendly}.  
I componenti sono sempre organizzati secondo il pattern \textbf{MVC (Model-View-Controller)}, con la suddivisione in:
\begin{itemize}
    \item \textbf{Boundary} - Interfaccia utente, responsabile dell'interazione con l'utente finale.
    \item \textbf{Controller} - Gestione logica di business.
    \item \textbf{Model} - Gestione dei dati e accesso al database.
    \item \textbf{Service} - Servizi di sicurezza e autenticazione.
    \item \textbf{Database} - PostgreSQL.
\end{itemize}




\end{document}