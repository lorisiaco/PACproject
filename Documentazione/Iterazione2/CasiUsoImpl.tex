\subsection{Casi d'Uso Implementati }

In questa iterazione sono stati sviluppati i seguenti casi d’uso ritenuti prioritari per lo sviluppo di \textbf{Spendly}.

\begin{table}[h]
    \centering
    \begin{tabular}{|c|l|}
    \hline
    \textbf{ID} & \textbf{Titolo} \\ \hline
    UC4 & Crea alert di gruppo\\ \hline
    UC6 & Elimina alert \\ \hline
    UC12 & Accedi gruppo \\ \hline
    UC13 & Inserisci spesa \\ \hline
    UC14 & Elimina spesa \\ \hline
    UC15 & Modifica spesa \\ \hline
    UC16 & Visualizza spese \\ \hline
    UC17 & Ricalcolo debiti \\ \hline
    \end{tabular}
    \caption{Iterazione2}
\end{table}

\subsubsection{UC4: Creazione Alert di Gruppo}
\textbf{Attori}: Utente amministratore, Sistema.
\newline
\newline
\textbf{Descrizione}: L’amministratore di un gruppo può creare un alert per segnalare il raggiungimento di una soglia limite di spesa.
\newline
\newline
\textbf{Flusso degli eventi}:
\begin{enumerate}
    \item L’amministratore accede alla pagina del gruppo.
    \item Clicca su "Crea Alert".
    \item Inserisce il limite di spesa e una descrizione opzionale.
    \item Il sistema salva l’alert e lo associa al gruppo.
\end{enumerate}

\subsubsection{UC6: Eliminazione Alert}
\textbf{Attori}: Utente amministratore, Sistema.
\newline
\newline
\textbf{Descrizione}: L’amministratore di un gruppo può eliminare un alert impostato in precedenza.
\newline
\newline
\textbf{Flusso degli eventi}:
\begin{enumerate}
    \item L’amministratore accede alla lista degli alert del gruppo.
    \item Seleziona un alert e clicca su "Elimina".
    \item Il sistema chiede conferma e poi rimuove l’alert.
\end{enumerate}

\subsubsection{UC13: Inserimento Spesa}
\textbf{Attori}: Utente, Sistema.
\newline
\newline
\textbf{Descrizione}: Un utente appartenente a un gruppo può aggiungere una spesa condivisa.
\newline
\newline
\textbf{Flusso degli eventi}:
\begin{enumerate}
    \item L’utente accede alla pagina del gruppo.
    \item Clicca su "Aggiungi Spesa".
    \item Inserisce l’importo, la descrizione e seleziona i partecipanti.
    \item Il sistema registra la spesa e aggiorna il bilancio del gruppo.
\end{enumerate}

\subsubsection{UC14: Eliminazione Spesa}
\textbf{Attori}: Utente, Sistema.
\newline
\newline
\textbf{Descrizione}: Un utente può eliminare una spesa precedentemente inserita.
\newline
\newline
\textbf{Flusso degli eventi}:
\begin{enumerate}
    \item L’utente accede alla lista delle spese del gruppo.
    \item Seleziona una spesa e clicca su "Elimina".
    \item Il sistema chiede conferma e poi rimuove la spesa dal gruppo.
\end{enumerate}

\subsubsection{UC15: Modifica Spesa}
\textbf{Attori}: Utente, Sistema.
\newline
\newline
\textbf{Descrizione}: Un utente può modificare i dettagli di una spesa condivisa nel gruppo.
\newline
\newline
\textbf{Flusso degli eventi}:
\begin{enumerate}
    \item L’utente accede alla lista delle spese.
    \item Seleziona una spesa e clicca su "Modifica".
    \item Modifica i dati della spesa (importo, descrizione, partecipanti).
    \item Il sistema aggiorna la spesa e ricalcola i bilanci.
\end{enumerate}

\subsubsection{UC16: Visualizzazione Spese}
\textbf{Attori}: Utente, Sistema.
\newline
\newline
\textbf{Descrizione}: Un utente può visualizzare tutte le spese del gruppo di cui fa parte.
\newline
\newline
\textbf{Flusso degli eventi}:
\begin{enumerate}
    \item L’utente accede alla pagina del gruppo.
    \item Clicca su "Visualizza Spese".
    \item Il sistema mostra l’elenco delle spese con dettagli su importo, data e partecipanti.
\end{enumerate}

\subsubsection{UC17: Ricalcolo Debiti}
\textbf{Attori}: Utente, Sistema.
\newline
\newline
\textbf{Descrizione}: Un utente può calcolare il riepilogo dei debiti tra i membri del gruppo.
\newline
\newline
\textbf{Flusso degli eventi}:
\begin{enumerate}
    \item L’utente accede alla pagina del gruppo.
    \item Clicca su "Calcola Debiti".
    \item Il sistema analizza le spese e genera un riepilogo dei debiti tra i membri.
    \item L’utente visualizza il riepilogo e le modalità di saldo consigliate.
\end{enumerate}


