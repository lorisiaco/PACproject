\subsection{UML Class Diagram }

\begin{center}
    \includegraphics[scale=0.15]{images/ClassDiagram.png}
\end{center}
\textbf{LoginDto $\rightarrow$ AppUser}:
\begin{itemize}
    \item Rappresenta il processo di autenticazione nel sistema.
    \item La relazione mostra che la classe \texttt{LoginDto} contiene i dati necessari (username e password) per verificare l'identità di un utente già registrato (\texttt{AppUser}).
    \item In altre parole, \texttt{LoginDto} fornisce un mezzo per confermare che un utente (\texttt{AppUser}) ha accesso al sistema.
\end{itemize}
\textbf{RegisterDto $\rightarrow$ AppUser}:
\begin{itemize}
    \item Rappresenta il processo di registrazione di un nuovo utente.
    \item La relazione indica che la classe \texttt{RegisterDto} contiene i dati necessari (nome, cognome, email, password, ecc.) per creare un nuovo account utente (\texttt{AppUser}) all'interno del sistema.
    \item Questo significa che la classe \texttt{RegisterDto} viene utilizzata per tradurre i dati di registrazione in un'istanza valida della classe \texttt{AppUser}.
\end{itemize}
\textbf{Group $\rightarrow$ AppUser (admin)}:
\begin{itemize}
    \item Rappresenta il ruolo di amministratore per un gruppo.
    \item Ogni gruppo (\texttt{Group}) ha un amministratore specifico (\texttt{AppUser}) che è responsabile della gestione del gruppo.
    \item La relazione è uno-a-uno (1:1), il che significa che ogni gruppo ha un solo amministratore, ma un amministratore può gestire più gruppi.
    \item Questo legame indica che l'amministratore è una figura centrale per la gestione delle attività e dei membri del gruppo.
\end{itemize}
\textbf{AppUser $\leftrightarrow$ Group (membri)}:
\begin{itemize}
    \item Rappresenta i membri di un gruppo e la loro relazione con i gruppi.
    \item La relazione è molti-a-molti ($m:n$), il che significa che:
    \begin{itemize}
        \item Ogni utente (\texttt{AppUser}) può essere membro di più gruppi (\texttt{Group}).
        \item Allo stesso tempo, ogni gruppo può avere più utenti come membri.
    \end{itemize}
    \item Questo legame mostra che l'utente e il gruppo sono fortemente interconnessi, poiché i gruppi esistono per aggregare utenti, e gli utenti possono partecipare a più attività o comunità rappresentate dai gruppi.
\end{itemize}


