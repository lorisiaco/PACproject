\subsection{Casi d'Uso Implementati }

In questa iterazione sono stati sviluppati i seguenti casi d’uso ritenuti prioritari per lo sviluppo di \textbf{Spendly}.

\begin{table}[h]
    \centering
    \begin{tabular}{|c|l|}
    \hline
    \textbf{ID} & \textbf{Titolo} \\ \hline
    UC1 & Login\\ \hline
    UC2 & Registrazione \\ \hline
    UC3 & Logout \\ \hline
    UC7 & Crea gruppo \\ \hline
    UC8 & Invita memebri \\ \hline
    UC9 & Elimina membri \\ \hline
    UC10 & Modica memebri \\ \hline
    UC11 & Elimina gruppo \\ \hline
    UC12 & Accedi gruppo \\ \hline
    \end{tabular}
    \caption{Iterazione1}
\end{table}

\subsubsection{UC1: Login}
\textbf{Attori}: Utente, Sistema.

\textbf{Descrizione}: L'utente può autenticarsi nel sistema per accedere al proprio account.

\textbf{Flusso degli eventi}:
\begin{enumerate}
    \item L'utente accede alla pagina di login.
    \item Inserisce email e password.
    \item Il sistema verifica le credenziali e autentica l'utente.
    \item L'utente viene reindirizzato alla dashboard.
\end{enumerate}

\subsubsection{UC2: Registrazione}
\textbf{Attori}: Utente, Sistema.

\textbf{Descrizione}: Un nuovo utente può registrarsi creando un account.

\textbf{Flusso degli eventi}:
\begin{enumerate}
    \item L'utente accede alla pagina di registrazione.
    \item Inserisce nome, email e password.
    \item Il sistema verifica che l’email non sia già registrata.
    \item Se la verifica è superata, il sistema crea l’account e lo memorizza.
    \item L'utente viene reindirizzato alla dashboard.
\end{enumerate}

\subsubsection{UC3: Logout}
\textbf{Attori}: Utente, Sistema.

\textbf{Descrizione}: L'utente può terminare la sessione ed effettuare il logout.

\textbf{Flusso degli eventi}:
\begin{enumerate}
    \item L'utente clicca su "Logout".
    \item Il sistema invalida la sessione e mostra la schermata di login.
\end{enumerate}

\subsubsection{UC7: Creazione Gruppo}
\textbf{Attori}: Utente amministratore, Sistema.

\textbf{Descrizione}: L’utente può creare un nuovo gruppo di spese per la condivisione con altri membri.

\textbf{Flusso degli eventi}:
\begin{enumerate}
    \item L’utente clicca su "Crea Gruppo".
    \item Inserisce il nome del gruppo e una descrizione opzionale.
    \item Il sistema crea il gruppo e assegna l’utente come amministratore.
    \item L'utente viene reindirizzato alla pagina del gruppo.
\end{enumerate}

\subsubsection{UC8: Invita Membri}
\textbf{Attori}: Utente amministratore, Sistema.

\textbf{Descrizione}: L’amministratore di un gruppo può invitare altri utenti a unirsi al gruppo di spese.

\textbf{Flusso degli eventi}:
\begin{enumerate}
    \item L’amministratore accede alla pagina del gruppo.
    \item Clicca su "Invita Membri" e inserisce l’email degli utenti da invitare.
    \item Il sistema invia un’email con l’invito e memorizza la richiesta.
    \item Gli utenti ricevono l’invito e possono accettarlo per entrare nel gruppo.
\end{enumerate}

\subsubsection{UC9: Elimina Membri}
\textbf{Attori}: Utente amministratore, Sistema.

\textbf{Descrizione}: L’amministratore di un gruppo può rimuovere un membro dal gruppo.

\textbf{Flusso degli eventi}:
\begin{enumerate}
    \item L’amministratore accede alla lista dei membri del gruppo.
    \item Seleziona il membro da rimuovere e clicca su "Elimina".
    \item Il sistema rimuove il membro e aggiorna la lista.
\end{enumerate}

\subsubsection{UC10: Modifica Membri}
\textbf{Attori}: Utente amministratore, Sistema.

\textbf{Descrizione}: L’amministratore di un gruppo può modificare i dettagli dei membri (ad esempio assegnare nuovi ruoli).

\textbf{Flusso degli eventi}:
\begin{enumerate}
    \item L’amministratore accede alla lista dei membri.
    \item Seleziona un membro e modifica i dettagli (es. ruolo nel gruppo).
    \item Il sistema aggiorna i dati e notifica il cambiamento.
\end{enumerate}

\subsubsection{UC11: Eliminazione Gruppo}
\textbf{Attori}: Utente amministratore, Sistema.

\textbf{Descrizione}: L’amministratore di un gruppo può eliminare definitivamente un gruppo di spese.

\textbf{Flusso degli eventi}:
\begin{enumerate}
    \item L’amministratore accede alle impostazioni del gruppo.
    \item Clicca su "Elimina Gruppo".
    \item Il sistema chiede conferma prima di procedere.
    \item Se confermato, il gruppo e tutte le sue spese vengono eliminate.
\end{enumerate}

\subsubsection{UC12: Accesso a un Gruppo}
\textbf{Attori}: Utente, Sistema.

\textbf{Descrizione}: Un utente può accedere a un gruppo di spese a cui è stato invitato.

\textbf{Flusso degli eventi}:
\begin{enumerate}
    \item L'utente riceve un invito via email o notifica nell’app.
    \item Clicca sul link di invito e accede alla web-app.
    \item Il sistema verifica la validità dell’invito e aggiunge l’utente al gruppo.
    \item L'utente viene reindirizzato alla pagina del gruppo.
\end{enumerate}