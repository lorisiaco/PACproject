\subsection{UML Component Diagram}


\begin{center}
    \includegraphics[scale=0.45]{images/ComponentDiagramV1.1.png}

\end{center}

Partendo dai casi d’uso selezionati per questa iterazione e procedendo con l’utilizzo delle euristiche di design, è stato possibile progettare l’architettura software del sistema \textbf{Spendly}.  
I componenti sono organizzati secondo il pattern \textbf{MVC (Model-View-Controller)}, con la suddivisione in:
\begin{itemize}
    \item \textbf{Boundary} - Interfaccia utente, responsabile dell'interazione con l'utente finale.
    \item \textbf{Controller} - Gestione logica di business.
    \item \textbf{Model} - Gestione dei dati e accesso al database.
    \item \textbf{Service} - Servizi di sicurezza e autenticazione.
    \item \textbf{Database} - PostgreSQL.
\end{itemize}
