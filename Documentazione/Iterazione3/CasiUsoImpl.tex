\subsubsection{Iterazione 3}
In questa iterazione sono stati sviluppati i casi d’uso relativi alle spese personali, alla gestione del budget e alla gestione del risparmio. Quest’ultima funzionalità permette all’utente di creare uno o più piani di risparmio con un obiettivo definito, in modo da mettere da parte denaro per esigenze future.

\begin{table}[h]
    \centering
    \begin{tabular}{|c|l|}
    \hline
    \textbf{ID} & \textbf{Titolo} \\ \hline
    UC19 & Inserisci spesa personale \\ \hline
    UC20 & Elimina spesa personale \\ \hline
    UC21 & Modifica spesa personale \\ \hline
    UC23 & Visualizza spese personali \\ \hline
    UC24 & Aggiungi denaro al budget \\ \hline
    UC26 & Visualizza budget \\ \hline
    UC27 & Paga spesa personale \\ \hline
    UC28 & Crea risparmio \\ \hline
    UC29 & Visualizza risparmio \\ \hline
    UC30 & Modifica risparmio \\ \hline
    UC31 & Elimina risparmio \\ \hline
    \end{tabular}
    \caption{Iterazione 3 - Casi d’Uso}
\end{table}

\subsubsection{UC19: Inserimento Spesa Personale}
\textbf{Attori}: Utente, Sistema. \\
\textbf{Descrizione}: Un utente può aggiungere una spesa personale.

\textbf{Flusso degli eventi}:
\begin{enumerate}
    \item L’utente accede alla sezione delle spese personali.
    \item Clicca su ``Aggiungi Spesa''.
    \item Inserisce l’importo, la descrizione e la data.
    \item Il sistema registra la spesa e aggiorna il budget.
\end{enumerate}

\subsubsection{UC20: Eliminazione Spesa Personale}
\textbf{Attori}: Utente, Sistema. \\
\textbf{Descrizione}: Un utente può eliminare una spesa personale.

\textbf{Flusso degli eventi}:
\begin{enumerate}
    \item L’utente accede alla lista delle spese personali.
    \item Seleziona una spesa e clicca su ``Elimina''.
    \item Il sistema chiede conferma e rimuove la spesa.
\end{enumerate}

\subsubsection{UC21: Modifica Spesa Personale}
\textbf{Attori}: Utente, Sistema. \\
\textbf{Descrizione}: Un utente può modificare i dettagli di una spesa personale.

\textbf{Flusso degli eventi}:
\begin{enumerate}
    \item L’utente accede alla lista delle spese personali.
    \item Seleziona una spesa e clicca su ``Modifica''.
    \item Modifica i dati (importo, descrizione, data).
    \item Il sistema aggiorna la spesa.
\end{enumerate}

\subsubsection{UC23: Visualizzazione Spese Personali}
\textbf{Attori}: Utente, Sistema. \\
\textbf{Descrizione}: Un utente può visualizzare tutte le proprie spese personali.

\textbf{Flusso degli eventi}:
\begin{enumerate}
    \item L’utente accede alla sezione delle spese personali.
    \item Clicca su ``Visualizza Spese''.
    \item Il sistema mostra l’elenco delle spese con dettagli.
\end{enumerate}

\subsubsection{UC27: Pagamento Spesa Personale}
\textbf{Attori}: Utente, Sistema. \\
\textbf{Descrizione}: Un utente può saldare una spesa personale direttamente tramite i metodi di pagamento disponibili.

\textbf{Flusso degli eventi}:
\begin{enumerate}
    \item L’utente accede alla lista delle spese personali.
    \item Seleziona una spesa e clicca su ``Paga''.
    \item Il sistema elabora il pagamento e aggiorna il budget scalando l’importo.
\end{enumerate}

\subsubsection{UC24: Aggiunta Denaro al Budget}
\textbf{Attori}: Utente, Sistema. \\
\textbf{Descrizione}: Un utente può aggiungere denaro al budget disponibile.

\textbf{Flusso degli eventi}:
\begin{enumerate}
    \item L’utente accede alla sezione budget.
    \item Clicca su ``Aggiungi Denaro''.
    \item Inserisce l’importo.
    \item Il sistema aggiorna il budget disponibile.
\end{enumerate}

\subsubsection{UC26: Visualizzazione Budget}
\textbf{Attori}: Utente, Sistema. \\
\textbf{Descrizione}: Un utente può visualizzare il budget disponibile e le spese totali.

\textbf{Flusso degli eventi}:
\begin{enumerate}
    \item L’utente accede alla sezione budget.
    \item Il sistema mostra l’importo disponibile e le spese sostenute.
\end{enumerate}

\subsubsection{UC28: Creazione Risparmio}
\textbf{Attori}: Utente, Sistema. \\
\textbf{Descrizione}: Un utente può creare un piano di risparmio con un obiettivo definito.

\textbf{Flusso degli eventi}:
\begin{enumerate}
    \item L’utente accede alla sezione risparmi.
    \item Clicca su ``Crea Risparmio''.
    \item Inserisce l’obiettivo e l’importo.
    \item Il sistema registra il piano e aggiorna i dati.
\end{enumerate}

\subsubsection{UC29: Visualizzazione Risparmio}
\textbf{Attori}: Utente, Sistema. \\
\textbf{Descrizione}: Un utente può visualizzare i piani di risparmio esistenti.

\textbf{Flusso degli eventi}:
\begin{enumerate}
    \item L’utente accede alla sezione risparmi.
    \item Il sistema mostra l’elenco dei risparmi con dettagli.
\end{enumerate}

\subsubsection{UC30: Modifica Risparmio}
\textbf{Attori}: Utente, Sistema. \\
\textbf{Descrizione}: Un utente può modificare i dettagli di un piano di risparmio.

\textbf{Flusso degli eventi}:
\begin{enumerate}
    \item L’utente accede alla lista dei risparmi.
    \item Seleziona un piano e clicca su ``Modifica''.
    \item Modifica l’importo o la descrizione.
    \item Il sistema aggiorna le informazioni.
\end{enumerate}

\subsubsection{UC31: Eliminazione Risparmio}
\textbf{Attori}: Utente, Sistema. \\
\textbf{Descrizione}: Un utente può eliminare un piano di risparmio.

\textbf{Flusso degli eventi}:
\begin{enumerate}
    \item L’utente accede alla lista dei risparmi.
    \item Seleziona un risparmio e clicca su ``Elimina''.
    \item Il sistema chiede conferma e lo rimuove tornando i soldi allocati al budget.
\end{enumerate}

