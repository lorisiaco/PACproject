\subsubsection{Casi d'uso Implementati}
% Richiamo al testo introduttivo generale già presente in Introduzione.tex.
% Di seguito sono riportati i casi d’uso specifici implementati in questa iterazione.

\textbf{Nota:} In Iterazione3, si è mantenuto il numero dei casi d’uso inerenti alle spese personali e al budget (da UC20 a UC28 come in CasiDuso.tex) e sono stati aggiunti i casi di gestione del risparmio, numerati da UC29 a UC32, in quanto non presenti in Iterazione0.

\begin{table}[h]
    \centering
    \begin{tabular}{|c|l|}
    \hline
    \textbf{ID} & \textbf{Titolo} \\ \hline
    UC20 & Inserisci spesa personale \\ \hline
    UC21 & Elimina spesa personale \\ \hline
    UC22 & Modifica spesa personale \\ \hline
    UC23 & Scegli tipologia spesa personale \\ \hline
    UC24 & Paga spesa personale \\ \hline
    UC25 & Visualizza spesa personale \\ \hline
    UC26 & Aggiungi denaro a budget \\ \hline
    UC27 & Togli denaro da budget \\ \hline
    UC28 & Visualizza budget \\ \hline
    UC29 & Crea risparmio \\ \hline
    UC30 & Visualizza risparmio \\ \hline
    UC31 & Modifica risparmio \\ \hline
    UC32 & Elimina risparmio \\ \hline
    \end{tabular}
    \caption{Iterazione 3 - Casi d’Uso Implementati.}
\end{table}

% Descrizioni dei casi d'uso
\subsubsection{UC20: Inserimento Spesa Personale}
\textbf{Attori}: Utente, Sistema. \\
\textbf{Descrizione}: Un utente può aggiungere una spesa personale. \\
\textbf{Flusso degli eventi}:
\begin{enumerate}
    \item L’utente accede alla sezione delle spese personali.
    \item Clicca su ``Aggiungi Spesa''.
    \item Inserisce l’importo, la descrizione e la data.
    \item Il sistema registra la spesa e aggiorna il budget.
\end{enumerate}

\subsubsection{UC21: Eliminazione Spesa Personale}
\textbf{Attori}: Utente, Sistema. \\
\textbf{Descrizione}: Un utente può eliminare una spesa personale. \\
\textbf{Flusso degli eventi}:
\begin{enumerate}
    \item L’utente accede alla lista delle spese personali.
    \item Seleziona una spesa e clicca su ``Elimina''.
    \item Il sistema chiede conferma e rimuove la spesa.
\end{enumerate}

\subsubsection{UC22: Modifica Spesa Personale}
\textbf{Attori}: Utente, Sistema. \\
\textbf{Descrizione}: Un utente può modificare i dettagli di una spesa personale. \\
\textbf{Flusso degli eventi}:
\begin{enumerate}
    \item L’utente accede alla lista delle spese personali.
    \item Seleziona una spesa e clicca su ``Modifica''.
    \item Modifica i dati (importo, descrizione, data).
    \item Il sistema aggiorna la spesa.
\end{enumerate}

\subsubsection{UC23: Scegli Tipologia Spesa Personale}
\textbf{Attori}: Utente, Sistema. \\
\textbf{Descrizione}: Un utente può scegliere la tipologia di una spesa personale. \\
\textbf{Flusso degli eventi}:
\begin{enumerate}
    \item L’utente accede alla sezione delle spese personali.
    \item Clicca su ``Scegli Tipologia''.
    \item Seleziona la tipologia della spesa.
    \item Il sistema aggiorna la tipologia della spesa.
\end{enumerate}

\subsubsection{UC24: Pagamento Spesa Personale}
\textbf{Attori}: Utente, Sistema. \\
\textbf{Descrizione}: Un utente può saldare una spesa personale direttamente tramite i metodi di pagamento disponibili. \\
\textbf{Flusso degli eventi}:
\begin{enumerate}
    \item L’utente accede alla lista delle spese personali.
    \item Seleziona una spesa e clicca su ``Paga''.
    \item Il sistema elabora il pagamento e aggiorna il budget scalando l’importo.
\end{enumerate}

\subsubsection{UC25: Visualizzazione Spesa Personale}
\textbf{Attori}: Utente, Sistema. \\
\textbf{Descrizione}: Un utente può visualizzare tutte le proprie spese personali. \\
\textbf{Flusso degli eventi}:
\begin{enumerate}
    \item L’utente accede alla sezione delle spese personali.
    \item Clicca su ``Visualizza Spese''.
    \item Il sistema mostra l’elenco delle spese con dettagli.
\end{enumerate}

\subsubsection{UC26: Aggiunta Denaro a Budget}
\textbf{Attori}: Utente, Sistema. \\
\textbf{Descrizione}: Un utente può aggiungere denaro al budget disponibile. \\
\textbf{Flusso degli eventi}:
\begin{enumerate}
    \item L’utente accede alla sezione budget.
    \item Clicca su ``Aggiungi Denaro''.
    \item Inserisce l’importo.
    \item Il sistema aggiorna il budget disponibile.
\end{enumerate}

\subsubsection{UC27: Togli Denaro da Budget}
\textbf{Attori}: Utente, Sistema. \\
\textbf{Descrizione}: Un utente può togliere denaro dal budget disponibile. \\
\textbf{Flusso degli eventi}:
\begin{enumerate}
    \item L’utente accede alla sezione budget.
    \item Clicca su ``Togli Denaro''.
    \item Inserisce l’importo.
    \item Il sistema aggiorna il budget disponibile.
\end{enumerate}

\subsubsection{UC28: Visualizzazione Budget}
\textbf{Attori}: Utente, Sistema. \\
\textbf{Descrizione}: Un utente può visualizzare il budget disponibile e le spese totali. \\
\textbf{Flusso degli eventi}:
\begin{enumerate}
    \item L’utente accede alla sezione budget.
    \item Il sistema mostra l’importo disponibile e le spese sostenute.
\end{enumerate}

\subsubsection{UC29: Creazione Risparmio}
\textbf{Attori}: Utente, Sistema. \\
\textbf{Descrizione}: Un utente può creare un piano di risparmio con un obiettivo definito. \\
\textbf{Flusso degli eventi}:
\begin{enumerate}
    \item L’utente accede alla sezione risparmi.
    \item Clicca su ``Crea Risparmio''.
    \item Inserisce l’obiettivo e l’importo.
    \item Il sistema registra il piano e aggiorna i dati.
\end{enumerate}

\subsubsection{UC30: Visualizzazione Risparmio}
\textbf{Attori}: Utente, Sistema. \\
\textbf{Descrizione}: Un utente può visualizzare i piani di risparmio esistenti. \\
\textbf{Flusso degli eventi}:
\begin{enumerate}
    \item L’utente accede alla sezione risparmi.
    \item Il sistema mostra l’elenco dei risparmi con dettagli.
\end{enumerate}

\subsubsection{UC31: Modifica Risparmio}
\textbf{Attori}: Utente, Sistema. \\
\textbf{Descrizione}: Un utente può modificare i dettagli di un piano di risparmio. \\
\textbf{Flusso degli eventi}:
\begin{enumerate}
    \item L’utente accede alla lista dei risparmi.
    \item Seleziona un piano e clicca su ``Modifica''.
    \item Modifica l’importo o la descrizione.
    \item Il sistema aggiorna le informazioni.
\end{enumerate}

\subsubsection{UC32: Elimina Risparmio}
\textbf{Attori}: Utente, Sistema. \\
\textbf{Descrizione}: Un utente può eliminare un piano di risparmio, restituendo l’importo relativo al budget. \\
\textbf{Flusso degli eventi}:
\begin{enumerate}
    \item L’utente accede alla lista dei risparmi.
    \item Seleziona un risparmio e clicca su ``Elimina''.
    \item Il sistema chiede conferma e, se confermato, rimuove il risparmio restaurando l’importo al budget.
\end{enumerate}
