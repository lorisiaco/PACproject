\subsection{Requisiti}
Il sistema deve soddisfare i seguenti requisiti funzionali e non funzionali:

\begin{itemize}
    \item \textbf{Requisiti Funzionali}:
    \begin{itemize}
        \item Gli utenti devono poter registrarsi e autenticarsi al sistema.
        \item Gli utenti devono poter aggiungere, modificare e cancellare le proprie spese.
        \item Gli utenti devono poter creare gruppi e gestire le spese condivise.
        \item Il sistema deve calcolare automaticamente il bilancio dei debiti tra i membri del gruppo.
        \item Il sistema deve inviare notifiche o alert agli utenti quando il loro budget per una determinata categoria di spesa viene superato.
        \item Gli utenti devono poter impostare budget personalizzati per categorie specifiche (es. alimentari, trasporti, svago, ecc.).
    \end{itemize}
    
    \item \textbf{Requisiti Non Funzionali}:
    \begin{itemize}
        \item L'applicazione deve essere intuitiva e facile da usare.
        \item Il sistema deve garantire la sicurezza dei dati personali degli utenti mediante crittografia e autenticazione sicura.
        \item Le transazioni e le modifiche devono essere sincronizzate in tempo reale tra tutti i dispositivi degli utenti.
        \item L’applicazione deve essere accessibile da diversi dispositivi (smartphone, tablet, desktop).
        \item Il sistema deve garantire elevate prestazioni, assicurando una risposta rapida alle richieste degli utenti.
    \end{itemize}
\end{itemize}
