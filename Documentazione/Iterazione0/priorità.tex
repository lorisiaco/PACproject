\subsection{Priorità casi d'uso}
    I casi d'uso possono essere suddivisi in tre categorie, a seconda della loro priorità nel processo di sviluppo:
    \begin{itemize}
        \item \textbf{Coda ad alta priorità:} 
        Contiene i requisiti essenziali per il corretto funzionamento dell'applicativo. Questi includono la creazione dei profili utente, la creazione di nuovi gruppi spese, con le funzionalità annesse(invita,elimina membri ecc ecc) ed infine la gestione spesa con l'algoritmo di calcolo debiti.
    
        \item \textbf{Coda a media priorità:} 
        Questa coda include funzionalità di supporto, principalmente orientate alla gestione spese personali.
    
        \item \textbf{Coda a bassa priorità:} 
        Contiene le funzionalità meno rilevanti ossia la tematica del budget , per le quali non è prevista una loro implementazione immediata. Tuttavia, potrebbero essere implementate in futuro, a seconda dell'andamento del progetto.
    \end{itemize}

    \begin{table}[H]
        \centering
        \begin{tabular}{|c|l|}
        \hline
        \textbf{ID} & \textbf{Titolo} \\ \hline
        UC1 & Login\\ \hline
        UC2 & Registrazione \\ \hline
        UC3 & Logout \\ \hline
        UC8 & Invita memebri \\ \hline
        UC9 & Elimina membri \\ \hline
        UC10 & Modica memebri \\ \hline
        UC11 & Elimina gruppo \\ \hline
        UC12 & Accedi gruppo \\ \hline
        UC13 & Inserisci spesa \\ \hline
        UC14 & Elimina spesa \\ \hline
        UC15 & Modifica spese \\ \hline
        UC16 & Visualizza spese \\ \hline
        UC17 & Ricalcolo debiti \\ \hline
        \end{tabular}
        \caption{Coda alta priorità}
    \end{table}

    \begin{table}[H]
        \centering
        \begin{tabular}{|c|l|}
        \hline
        \textbf{ID} & \textbf{Titolo} \\ \hline
        UC4 & Crea alert di gruppo\\ \hline
        UC5 & Modica alert \\ \hline
        UC6 & Elimina alert \\ \hline
        UC18 & Inserisci spesa personale \\ \hline
        UC19 & Elimina spesa personale \\ \hline
        UC20 & Modica spesa personale \\ \hline
        UC21 & Visualizza spesa personale \\ \hline
        \end{tabular}
        \caption{Coda media priorità}
    \end{table}

    \begin{table}[H]
        \centering
        \begin{tabular}{|c|l|}
        \hline
        \textbf{ID} & \textbf{Titolo} \\ \hline
        UC22 & Inserisci budget\\ \hline
        UC23 & Elimina budget \\ \hline
        UC24 & Modica budget \\ \hline
        UC25 & Visualizza budget \\ \hline
        \end{tabular}
        \caption{Coda bassa priorità}
    \end{table}
\textbf{Perché si dividono in code di priorità i casi d'uso?}
\\
\\
In quanto lo sviluppo del progetto viene sviluppato tramite il processo Agile Model-Driven Developmen.
\\
AMDD combina i principi della modellazione guidata dallo sviluppo con la flessibilità dei metodi agili, che a differenza dello sviluppo tradizionale, in cui tutta l’analisi viene completata prima di iniziare la programmazione, prevede una modellazione iniziale leggera e iterativa, concentrata solo sugli aspetti essenziali per l’iterazione corrente.
\newline
Ad ogni ciclo di sviluppo, il team seleziona un numero limitato di casi d’uso dalla coda di priorità più alta, lavorando su di essi fino a ottenere una funzionalità completa e testata. Questo approccio:
\begin{itemize}
    \item Riduce il rischio affrontando le funzionalità critiche prima possibile.
    \item Favorisce il feedback continuo, permettendo agli stakeholder di validare il progresso e suggerire miglioramenti.
    \item Adatta lo sviluppo alle necessità reali, permettendo di rivalutare le priorità in base all’andamento del progetto.
\end{itemize}
Grazie a questo metodo, il sistema viene costruito in modo solido e incrementale, con la possibilità di aggiungere nuove funzionalità man mano che il progetto evolve.