\subsection{Priorità casi d'uso}
I casi d'uso sono suddivisi in tre categorie, in linea con l'ordine stabilito in \textbf{Casi d'Uso}:

% Tabella per la Coda ad alta priorità
\begin{table}[H]
    \centering
    \begin{tabular}{|c|p{10cm}|}
        \hline
        \textbf{ID} & \textbf{Titolo} \\ \hline
        UC1  & Login \\ \hline
        UC2  & Registrazione \\ \hline
        UC3  & Logout \\ \hline
        UC8  & Crea gruppo \\ \hline
        UC9  & Invita membri \\ \hline
        UC10 & Modifica membri \\ \hline
        UC11 & Elimina gruppo \\ \hline
        UC13 & Accedi gruppo \\ \hline
        UC14 & Inserisci spesa \\ \hline
        UC15 & Elimina spesa \\ \hline
        UC16 & Modifica spesa \\ \hline
        UC17 & Scegli tipologia spesa \\ \hline
        UC18 & Visualizza spese \\ \hline
        UC19 & Ricalcolo debiti \\ \hline
    \end{tabular}
    \caption{Coda alta priorità}
\end{table}

% Tabella per la Coda a media priorità
\begin{table}[H]
   \centering
   \begin{tabular}{|c|p{10cm}|}
       \hline
       \textbf{ID} & \textbf{Titolo} \\ \hline
       UC4  & Crea alert di gruppo \\ \hline
       UC5  & Modifica alert \\ \hline
       UC6  & Elimina alert \\ \hline
       UC7  & Scegli tipologia alert \\ \hline
       UC20 & Inserisci spesa personale \\ \hline
       UC21 & Elimina spesa personale \\ \hline
       UC22 & Modifica spesa personale \\ \hline
       UC23 & Scegli tipologia spesa personale \\ \hline
       UC24 & Paga spesa personale \\ \hline
       UC25 & Visualizza spesa personale \\ \hline
   \end{tabular}
   \caption{Coda a media priorità}
\end{table}

% Tabella per la Coda a bassa priorità
\begin{table}[H]
   \centering
   \begin{tabular}{|c|p{10cm}|}
       \hline
       \textbf{ID} & \textbf{Titolo} \\ \hline
       UC26 & Aggiungi denaro a budget \\ \hline
       UC27 & Togli denaro da budget \\ \hline
       UC28 & Visualizza budget \\ \hline
   \end{tabular}
   \caption{Coda a bassa priorità}
\end{table}


\textbf{Perché si dividono in code di priorità i casi d'uso?}
\\
\\
In quanto lo sviluppo del progetto viene sviluppato tramite il processo Agile Model-Driven Development.
\\
AMDD combina i principi della modellazione guidata dallo sviluppo con la flessibilità dei metodi agili, che a differenza dello sviluppo tradizionale, in cui tutta l’analisi viene completata prima di iniziare la programmazione, prevede una modellazione iniziale leggera e iterativa, concentrata solo sugli aspetti essenziali per l’iterazione corrente.
\newline
Ad ogni ciclo di sviluppo, il team seleziona un numero limitato di casi d’uso dalla coda di priorità più alta, lavorando su di essi fino a ottenere una funzionalità completa e testata. Questo approccio:
\begin{itemize}
    \item Riduce il rischio affrontando le funzionalità critiche prima possibile.
    \item Favorisce il feedback continuo, permettendo agli stakeholder di validare il progresso e suggerire miglioramenti.
    \item Adatta lo sviluppo alle necessità reali, permettendo di rivalutare le priorità in base all’andamento del progetto.
\end{itemize}
Grazie a questo metodo, il sistema viene costruito in modo solido e incrementale, con la possibilità di aggiungere nuove funzionalità man mano che il progetto evolve.