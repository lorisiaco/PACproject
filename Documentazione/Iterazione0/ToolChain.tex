\subsection{Tool Chain}
Lo sviluppo del progetto è supportato dai seguenti tool:

\begin{itemize}
    \item \textbf{Eclipse}: IDE per lo sviluppo software utilizzato per la scrittura di tutto il codice \textbf{(Java)} del sistema. Alcuni degli altri tool utilizzati sono integrati con Eclipse.
    \item \textbf{Spring Boot}: framework Java per lo sviluppo di applicazioni web basate sui microservizi.
    \item \textbf{MongoDB Atlas}: cloud database non relazionale (i dati sono salvati come documenti JSON). Il cluster utilizzato si appoggia ad un server \textbf{AWS}.
    \item \textbf{JAutoDoc}: plugin di Eclipse per la generazione dei \textbf{Javadoc} che permettono di generare la documentazione del codice Java a partire dai commenti del codice.
    \item \textbf{JUnit 4}: framework per i test di unità in Java.
    \item \textbf{EclEmma}: plugin di Eclipse per la verifica della copertura del codice.
    \item \textbf{JGraphT}: libreria Java per la modellazione di grafi.
    \item \textbf{STAN4J}: software per l’analisi statica di progetti Java.
    \item \textbf{GitHub}: piattaforma per il versionamento basata su Git. È stato utilizzato anche GitHub Desktop che permette di interagire dal proprio PC con GitHub utilizzando una semplice GUI.
    \item \textbf{StarUML}: software per la creazione di modelli UML.
    \item \textbf{diagrams.net}: software online per la creazione di modelli in notazione libera.
\end{itemize}