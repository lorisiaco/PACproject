\subsection{Priorità casi d'uso}
I casi d'uso sono suddivisi in tre categorie, in linea con l'ordine stabilito in \textbf{Casi d'Uso}:

\begin{itemize}
    \item \textbf{Coda ad alta priorità:} 
    \begin{itemize}
        \item UC1 - Login
        \item UC2 - Registrazione
        \item UC3 - Logout
        \item UC7 - Crea gruppo
        \item UC8 - Invita membri
        \item UC9 - Elimina membri
        \item UC10 - Modifica membri
        \item UC11 - Elimina gruppo
        \item UC12 - Accedi gruppo
        \item UC13 - Inserisci spesa
        \item UC14 - Elimina spesa
        \item UC15 - Modifica spesa
        \item UC16 - Scegli tipologia spesa
        \item UC17 - Visualizza spese
        \item UC18 - Ricalcolo debiti
    \end{itemize}
    
    \item \textbf{Coda a media priorità:} 
    \begin{itemize}
        \item UC4 - Crea alert di gruppo
        \item UC5 - Modifica alert
        \item UC6 - Elimina alert
        \item UC19 - Inserisci spesa personale
        \item UC20 - Elimina spesa personale
        \item UC21 - Modifica spesa personale
        \item UC22 - Scegli tipologia spesa personale
        \item UC23 - Paga spesa personale
        \item UC24 - Visualizza spesa personale
    \end{itemize}
    
    \item \textbf{Coda a bassa priorità:} 
    \begin{itemize}
        \item UC25 - Aggiungi denaro a budget
        \item UC26 - Togli denaro da budget
        \item UC27 - Visualizza budget
    \end{itemize}
\end{itemize}

\textbf{Perché si dividono in code di priorità i casi d'uso?}
\\
\\
In quanto lo sviluppo del progetto viene sviluppato tramite il processo Agile Model-Driven Developmen.
\\
AMDD combina i principi della modellazione guidata dallo sviluppo con la flessibilità dei metodi agili, che a differenza dello sviluppo tradizionale, in cui tutta l’analisi viene completata prima di iniziare la programmazione, prevede una modellazione iniziale leggera e iterativa, concentrata solo sugli aspetti essenziali per l’iterazione corrente.
\newline
Ad ogni ciclo di sviluppo, il team seleziona un numero limitato di casi d’uso dalla coda di priorità più alta, lavorando su di essi fino a ottenere una funzionalità completa e testata. Questo approccio:
\begin{itemize}
    \item Riduce il rischio affrontando le funzionalità critiche prima possibile.
    \item Favorisce il feedback continuo, permettendo agli stakeholder di validare il progresso e suggerire miglioramenti.
    \item Adatta lo sviluppo alle necessità reali, permettendo di rivalutare le priorità in base all’andamento del progetto.
\end{itemize}
Grazie a questo metodo, il sistema viene costruito in modo solido e incrementale, con la possibilità di aggiungere nuove funzionalità man mano che il progetto evolve.