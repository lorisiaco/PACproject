\subsection{Architettura}
L'architettura del sistema è basata su un'architettura a microservizi con i seguenti componenti principali:
\begin{itemize}
    \item \textbf{Client/Frontend}: Implementato in React.js, il frontend fornisce un'interfaccia utente intuitiva per la gestione delle spese.
    \item \textbf{API Gateway}: Un punto di ingresso unico per tutte le richieste del client, responsabile dell'instradamento verso i microservizi corretti.
    \item \textbf{Backend (Microservizi)}: I microservizi sono sviluppati in Spring Boot e includono:
    \begin{itemize}
        \item \textbf{Servizio Utente}: Gestisce la registrazione, l'autenticazione e il profilo utente.
        \item \textbf{Servizio Gestione Spese}: Gestisce le operazioni CRUD sulle spese.
        \item \textbf{Servizio Gruppi}: Gestisce la creazione e la modifica dei gruppi.
    \end{itemize}
    \item \textbf{Database}: Utilizzo di PostgreSQL per la gestione dei dati relazionali.
\end{itemize}
