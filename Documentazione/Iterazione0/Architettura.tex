\subsection{Architettura}
L'architettura del sistema è basata su un'architettura MVC (Model View Controller) con le seguenti componenti principali:
\begin{itemize}
    \item \textbf{Frontend}: Implementato in HTML e CSS, il frontend fornisce un'interfaccia utente intuitiva per la gestione delle spese.
    \item \textbf{Database}: Utilizza PostgreSQL come sistema di gestione dei dati.
    \item \textbf{Backend}: Implementato in Spring Boot e include:
    \begin{itemize}
        \item \textbf{Servizio Utente}: Gestisce la registrazione, l'autenticazione e il profilo utente.
        \item \textbf{Servizio Gestione Spese}: Gestisce le operazioni CRUD sulle spese.
        \item \textbf{Servizio Gruppi}: Gestisce la creazione e la modifica dei gruppi.
    \end{itemize}
    
    \item \textbf{Flusso delle richieste}: 
    \begin{itemize}
        \item \textbf{} L'utente interagisce con il frontend (browser) tramite pagine HTML.
        \item \textbf{} Le richieste vengono inviate al backend (controller Spring Boot) attraverso HTTP.
        \item \textbf{} Il backend elabora la richiesta, utilizza i servizi per applicare la logica di business e accede al database.
        \item \textbf{}La risposta (dati o una nuova pagina HTML) viene inviata al frontend e presentata all'utente.
    \end{itemize}
\end{itemize}
